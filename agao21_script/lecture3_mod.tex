% \documentclass[draft,11pt]{article}
\chapter{Convexity and Second Derivatives,
  Gradient Descent and Acceleration}
\label{sec:gd}
%\allowdisplaybreaks

%%% for this lecture
%\newcommand\gap{\text{gap}}

%\begin{document}
\sloppy
%\lecture{3 --- Monday, March 4th}
%{Spring 2020}{Rasmus Kyng}{Convexity and Second Derivatives,
%  Gradient Descent and Acceleration}

%%% TODO comments from students
% 1)
% Taylor's thm 2nd order doesn't require TWICE continuously diff,
% according to wikipedia
% reason: e.g. 1st order also only needs 1st deriv to exist, not to be continuous
% ah, for a function to be differentiable, it can't have a jump
% discontinuity in the derivative, but it can have an essential
% discontinuity (fast oscillation)
% https://en.wikipedia.org/wiki/Classification_of_discontinuities
%
%
% 2)
% proof of sufficient condition for local extremum is not quite right


%In the lecture today, we didn't cover accelerated gradient
%descent. We'll do that next week instead.

\paragraph{Notation for this chapter.}
  In this chapter, we sometimes consider a multivariate functions $f$
  whose domain is a set $S \subseteq \R^n$, which we will require to
  be open.
  When we additionally require that $S$ is convex, we will specify this.
  Note that $S = \R^n$ is both open and convex and it suffices to keep
  this case in mind.
  Things sometimes get more complicated if $S$ is not open, e.g. when the
  domain of $f$ has a boundary.
  We will leave those complications for another time.

\section{A Review of Linear Algebra}
\paragraph{Semi-definiteness of a matrix.}  The following classification of symmetric matrices will be useful.

\boxdef{
    Let $\AA$ by a symmetric matrix in $\R^{n\times n}$. We say that $\AA$ is:
    \begin{enumerate}
        \item \emph{positive definite} iff $\xx^\trp \AA\xx>0$ for all $x\in\R^n\setminus\{0\}$;
        % \item \emph{negative definite} iff $\xx^\trp \AA\xx<0$ for all $x\in\R^n\setminus\{0\}$;
        \item \emph{positive semidefinite} iff $\xx^\trp \AA\xx\geq0$ for all $x\in\R^n$;
        % \item \emph{negative semidefinite} iff $\xx^\trp \AA\xx\leq 0$ for all $x\in\R^n$;
        \item If neither $\AA$ nor $-\AA$ is positive semi-definite, we say that $\AA$ is \emph{indefinite}.
    \end{enumerate}
}

\paragraph{Example: indefinite matrix.} Consider the following matrix $\AA$:
\begin{displaymath}
    \AA:=\begin{bmatrix}
        +4 & -1\\
        -1 & -2
    \end{bmatrix}
  \end{displaymath}
% Then we have $\xx^\trp  \AA \xx= 4x(1)_1^2 - 2x_1 x_2 - 2x_2^2$.
% For $\xx= (1\; 0)^\trp$, we have $\xx^\trp  \AA \xx= 4 >0$. For $\xx= (0\; 1)^\trp$ we have $\xx^\trp  \AA \xx= -2< 0$. $\AA$ is therefore indefinite.
For $\xx=
\begin{pmatrix}
  1 \\ 0
\end{pmatrix}$, we have $\xx^\trp  \AA \xx= 4 >0$. For $\xx=
\begin{pmatrix}
 0 \\ 1
\end{pmatrix}
$ we have $\xx^\trp  \AA \xx= -2< 0$. $\AA$ is therefore indefinite.

The following theorem gives a useful characterization of (semi)definite matrices.

\begin{theorem}
\label{thm:psdfromeigs}
    Let $\AA$ be a symmetric matrix in $\R^{n\times n}$.
    \begin{enumerate}
        \item $\AA$ is positive definite iff all its eigenvalues are positive;
        \item $\AA$ is positive semidefinite iff all its eigenvalues are non-negative;
    \end{enumerate}
\end{theorem}
In order to prove this theorem, let us first recall the Spectral
Theorem for symmetric matrices.
\begin{theorem}[The Spectral Theorem for Symmetric Matrices]
  \label{thm:spectral}


 For all symmetric $\AA \in \R^{n \times n}$ there exist $\VV \in \R^{n
  \times n}$ and a diagonal matrix $\LLambda \in \R^{n \times n}$
s.t.
\begin{enumerate}
\item $\AA = \VV \LLambda\VV^\trp$.
\item $\VV^\trp \VV = \II$ (the $n \times n$ identity matrix). I.e. the columns of
  $\VV$ form an orthonormal basis.
  Furthermore, $\vv_i$ is an eigenvector of
  $\lambda_i(\AA)$, the $i$th eigenvalue of $\AA$.
\item $\LLambda_{ii} = \lambda_i(\AA)$.
\end{enumerate}
\end{theorem}
Using the Spectral Theorem, we can show the following result:
\begin{theorem}[The Courant-Fischer Theorem]
  \label{thm:courant-fischer}
  Let $\AA$ be a symmetric matrix in $\R^{n\times n}$, with
  eigenvalues $\lambda_1\leq \lambda_2 \leq \ldots \leq \lambda_n$.
  Then
  \begin{enumerate}
  \item
  \label{thm:courant-fischer:minmax}
    \[
    \lambda_i = \min_{
      \substack{
        \text{subspace } W \subseteq \R^n
        \\
        \dim{W} = i
      }
    }
    \max_{
      \xx \in W, \xx \neq \veczero
    }
    \frac{\xx^\trp \AA\xx}{\xx^\trp\xx}
  \]
\item
  \label{thm:courant-fischer:maxmin}
    \[
    \lambda_i
    =
    \max_{
      \substack{
        \text{subspace } W \subseteq \R^n
        \\
        \dim{W} = n+1-i
      }
    }
    \min_{
      \xx \in W, \xx \neq \veczero
    }
    \frac{\xx^\trp \AA\xx}{\xx^\trp\xx}
    \]
  \end{enumerate}
  \end{theorem}
Theorem~\ref{thm:psdfromeigs} is an immediate corollary of
Theorem~\ref{thm:courant-fischer}, since we can see that the minimum value
of
the quadratic form $\xx^\trp \AA \xx$ over $\xx \in W = \R^n$ is
$\lambda_1(\AA)\norm{\xx}_2^2$.
\begin{proof}[Proof of Theorem~\ref{thm:courant-fischer}]
% The spectral theorem tells us that any real symmetric matrix is
% diagonalizable by an orthonormal matrix (given by the eigenvectors).

We start by showing Part~\ref{thm:courant-fischer:minmax}.

Consider letting $W = \Span\setof{\vv_1, \ldots,
  \vv_i}
$, and normalize $\xx \in W$ so that $\norm{\xx}_2 = 1$. Then $\xx = \sum_{j = 1}^i
\cc(j) \vv_j$ for some vector $\cc \in \R^i$ with $\norm{\cc}_2 = 1$.

Using the decomposition from Theorem~\ref{thm:spectral} $\AA=
\VV \LLambda\VV^\trp$ where $\LLambda$ is a diagonal matrix of
eigenvalues of $\AA$, which we take to be sorted in increasing order.
Then $\xx^\trp \AA\xx= \xx^\trp \VV^\trp
\LLambda\VV\xx= (\VV\xx)^\trp  \LLambda(\VV\xx) = \sum_{j = 1}^i
\lambda_j \cc(j)^2 \leq \lambda_i \norm{\cc}_2^2 = \lambda_i $.
So this choice of $W$ ensures the maximizer cannot achieve a value above $\lambda_i$.
% So ``maximizer'' can obtain a value of $\lambda_i$ given this $W$, by choosing $\cc =
% \ee_i$, the $i$th standard basis vector.
% At the same time, this $W$ ensures the maximizer cannot do better, as $\sum_{j = 1}^i
% \lambda_j \cc(j)^2 \leq \lambda_i \norm{\cc}_2^2 = \lambda_i $.

But is it possible that the ``minimizer'' can do better by choosing
a different $W$?
Let $T = \Span\setof{\vv_i, \ldots,\vv_n}$.
As $\dim{T} = n+1-i$ and $\dim{W} = i$, we must have $\dim{W
  \intersect T} \geq 1$, by a standard property of subspaces.
Hence for any $W$ of this dimension,
\begin{align*}
    \max_{
      \xx \in W, \xx \neq \veczero
    }
    \frac{\xx^\trp \AA\xx}{\xx^\trp\xx}
&\geq
    \max_{
      \xx \in W \intersect T , \xx \neq \veczero
    }
\frac{\xx^\trp \AA\xx}{\xx^\trp\xx}
\\
&\geq  \min_{
      \substack{
        \text{subspace } V \subseteq T
        \\
        \dim{V} = 1
      }
    }
    \max_{
      \xx \in V , \xx \neq \veczero
    }
  \frac{\xx^\trp \AA\xx}{\xx^\trp\xx}
= \lambda_i,
\end{align*}
where the last equality follows from a similar calculation to our
first one.
Thus, $\lambda_i$ can always be achieved by the ``maximizer'' for all
$W$ of this dimension.
% Let us define $\yy:=\VV\xx$,
% then we have:
% %
% $$\xx^\trp \AA\xx= \sum_{i=1}^n \lambda_iy_i^2$$
% %
% Consider letting $W = \Span\setof{\vv_1, \ldots,
%   \vv_i}
% $, normalize $\xx \in W$ so that $\norm{\xx}_2 = 1$ then $\xx = \sum_{j = 1}^i
% \cc(j) \vv_j$ for a vector $\cc_i$ with $\norm{\cc_i} = 1$, and
% where $\lambda_1,\dots,\lambda_n$ are the diagonal elements of
% $\LLambda$, the eigenvalues of $\AA$. Also note that since $\VV$ is
% invertible, $\yy=0$ iff $\xx=\veczero$.  If $\lambda_i> 0$ for all
% $i$, then we see that $\xx^\trp \AA\xx>0$ for all $\xx\neq
% \veczero$. For the other direction, choosing $\yy$ such that $y_i = 1$
% and $y_j = 0$ for $j\neq i$ yields $\lambda_i > 0$.


Part~\ref{thm:courant-fischer:maxmin} can be dealt with similarly.

\end{proof}

\paragraph{Example: a positive semidefinite matrix.} Consider the
following matrix $\AA$:

  \begin{displaymath}
    \AA:=\begin{bmatrix}
        1& -1\\
        -1 & 1
    \end{bmatrix}
  \end{displaymath}
For $\xx=
\begin{pmatrix}
  1 \\ 1
\end{pmatrix}$,
we have $ \AA \xx = \veczero$, so $\lambda = 0$ is an eigenvalue of
$\AA$.
For $\xx=
\begin{pmatrix}
  1 \\ -1
\end{pmatrix}$,
we have $ \AA \xx = \begin{pmatrix}
  2 \\ -2
\end{pmatrix}
= 2 \xx
$, so $\lambda = 2$ is the other eigenvalue of
$\AA$.
As both are non-negative, by the theorem above, $\AA$ is positive semidefinite.


Since we are learning about symmetric matrices, there is one more fact
that everyone should know about them. We'll use $\lambda_{\max}(\AA)$ denote
maximum eigenvalue of a matrix $\AA$, and $\lambda_{\min}(\AA)$ the minimum.
\begin{claim}
  \label{clm:symnorm}
  For a symmetric matrix
  $\AA \in \R^{n\times n}$,
  $\norm{\AA} =
  \max(\abs{\lambda_{\max}(\AA)},
  \abs{\lambda_{\min}(\AA)})
$.
\end{claim}




\section{Characterizations of Convexity and Optimality via Second Derivatives}
We will now use the second derivatives of a function to obtain
characterizations of convexity and optimality. We will begin by
introducing the \emph{Hessian}, the matrix of pairwise second
derivatives of a function.
We will see that it plays a role in
approximating a function via a second-order Taylor expansion.
We will then use \emph{semi-definiteness} of the Hessian matrix to characterize both conditions of optimality as well as the convexity of a function.

\boxdef{
Given a function $f:S \to \R$ its \textbf{Hessian} matrix at point $\xx \in S$ denoted $\HH_{f}(\xx)$ (also sometimes denoted $\grad^2 f(\xx)$) is:
\begin{displaymath}
    \HH_f(\xx) := \begin{bmatrix}
        \frac{\partial^2f(\xx)}{\partial \xx(1)^2} & \frac{\partial^2f(\xx)}{\partial \xx(1)\partial \xx(2)}&\dots&\frac{\partial^2f(\xx)}{\partial \xx(1)\partial \xx(n)}\\
        \frac{\partial^2f(\xx)}{\partial \xx(2)\partial \xx(1)} & \frac{\partial^2f(\xx)}{\partial \xx(2)^2}&\dots&\frac{\partial^2f(\xx)}{\partial \xx(2)\partial \xx(n)}\\
        \vdots&\vdots&\ddots&\vdots\\
        \frac{\partial^2f(\xx)}{\partial \xx(n)\partial \xx(1)}&\frac{\partial^2f(\xx)}{\partial \xx(n)\partial \xx(2)}&\dots&\frac{\partial^2f(\xx)}{\partial \xx(n)^2}
    \end{bmatrix}
\end{displaymath}
}



\paragraph{Second-order Taylor expansion.}  When $f$ is twice differentiable it is possible to obtain an approximation of $f$ by quadratic functions.
%Let $S \subseteq \R^n$ be an open set.
Our definition of $f:S\rightarrow \R$ being twice (Fr\'{e}chet) differentiable at
$\xx \in S$ is that there exists $\grad f(\xx) \in \R^n$ and
$\HH_f(\xx) \in R^{n \times n}$ s.t.
\begin{displaymath}
  \lim_{\ddelta \to \veczero}
  \frac{\norm{f(\xx+\ddelta) -f(\xx)
      -\left( \grad f(\xx)^{\trp} \ddelta + \frac{1}{2}\ddelta^\trp
        \HH_f(\xx) \ddelta \right) }_2}{
    \norm{\ddelta}_2^2
  } = 0
    .
\end{displaymath}
This is equivalent to saying that for all $\ddelta$
\begin{displaymath}
  %\forall \ddelta\in \R^n,
  f(\xx+\ddelta) = f(\xx) + \grad f(\xx)^\trp \ddelta + \frac{1}{2}\ddelta^\trp  \HH_f(\xx) \ddelta + o(\norm{\ddelta}_2^2).
\end{displaymath}
where by definition:
\begin{displaymath}
  \lim_{\ddelta \rightarrow \veczero} \frac{o(\norm{\ddelta}^2)}{\norm{\ddelta}_2^2} = 0
\end{displaymath}
% Similarly to the first-order alternative expansion, we have, when $f$ is of class $C^1$ over $[\bar{\xx}, \xx]$ and twice differentiable over this interval:
% \begin{displaymath}
% \forall \xx \in \R^n,    f(\xx) = f(\bar{\xx}) + \grad f(\bar{\xx})^\trp (\xx-\bar{\xx})+ \frac{1}{2}(\xx-\bar{\xx})^\trp  H_f(\zz(\xx-\bar{\xx})
% \quad\text{for some $\zz\in[\bar{\xx}, \xx]$.}
% \end{displaymath}
We say that $f$ is \emph{ continuously differentiable} on a set $S \subseteq \R^n $ if it is
twice differentiable and in addition the gradient and Hessian are continuous on $S$.

As for first order expansions, we have a Taylor's Theorem, which we
state in the so-called remainder form.
\begin{theorem}[Taylor's Theorem, multivariate second-order remainder
  form]
  \label{thm:taylorremainder2}
  If $f:S\to \R$ is twice continuously differentiable over $[\xx, \yy]$, then for
  some $\zz \in [\xx, \yy]$,
\begin{displaymath}
  f(\yy) = f(\xx) + \grad f(\xx)^\trp (\yy-\xx)
  + \frac{1}{2}(\yy-\xx)^\trp  \HH_f(\zz) (\yy-\xx)
\end{displaymath}
\end{theorem}

\subsection{A Necessary Condition for Local Extrema}

Recall that in the previous chapter, we show the following proposition.
\begin{proposition}
\label{prp:gradatmin}
    If $\xx$ is a local extremum of a differentiable function
    $f:S\rightarrow \R$
    then $\grad f(\xx) = \veczero$.
\end{proposition}

We can now give the second-order necessary conditions for local extrema via the Hessian.
\begin{theorem}
Let $f:S \rightarrow\R$ be a function twice differentiable at
    $\xx\in S$.
    If $\xx$ is a local minimum, then $\HH_f(\xx)$ is positive
    semidefinite.
    % If $\xx$ is a local maximum, then $-\HH_f(\xx)$ is positive
    % semidefinite.
\end{theorem}
\begin{proof}
  Let us assume that $\xx$ is a local minimum. We know from
    Proposition~\ref{prp:gradatmin} that $\grad f(\xx)= \veczero$, hence the second-order expansion at $\xx$ takes the form:
    \begin{displaymath}
        f(\xx + \lambda\dd) = f(\xx) + \lambda^2 \frac{1}{2} \dd^\trp
        \HH_f(\xx)\dd + o(\lambda^2 \norm{\dd}_2^2)
    \end{displaymath}
    Because $\xx$ is a local minimum,
    we can then derive
\begin{displaymath}
  0 \leq\lim_{\lambda \to 0^+}
  \frac{f(\xx+\lambda\dd)-f(\xx)}{\lambda^2} = \frac{1}{2}\dd^\trp \HH_f(\xx)\dd
\end{displaymath}
This is true for any $\dd$, hence $\HH_f(\xx)$ is positive semidefinite.
\end{proof}
  \begin{remark}
    Again, for this proposition to hold, it is important that $S$ is
    open.
  \end{remark}

\subsection{A sufficient condition for local extrema}
A local minimum thus is a stationary point and has a positive
semi-definite Hessian.
The converse is almost true, but we need to strengthen the
Hessian condition slightly.

\begin{theorem}
Let $f:S \rightarrow\R$ be a function twice differentiable at
    a stationary point $\xx\in S$.
     If $\HH_f(\xx)$ is positive definite then $\xx$ is a local minimum.
\end{theorem}

\begin{proof}
    Let us assume that $\HH_f(\xx)$ is positive definite. We know that $\xx$ is a stationary point. We can write the second-order expansion at $\xx$:
    \begin{displaymath}
        f(\xx +\ddelta) = f(\xx) + \frac{1}{2} \ddelta^\trp
        \HH_f(\xx) \ddelta + o( \norm{\ddelta}_2^2)
      \end{displaymath}
    Because the Hessian is positive definite, it has a strictly
    positive minimum eigenvalue $\lambda_{\min}$, we can conclude that
    $\ddelta^\trp \HH_f(\xx) \ddelta \geq \lambda_{\min} \norm{\ddelta}_2^2$.
    From this, we conclude that when $\norm{\ddelta}_2^2$ is small
    enough,
    $f(\xx+\ddelta)-f(\xx) \geq \frac{1}{4} \lambda_{\min}
    \norm{\ddelta}_2^2 > 0$.
   This proves that $\xx$ is a local minimum.
\end{proof}

\begin{remark}
    When $\HH_f(\xx)$ is indefinite at a stationary point $\xx$, we have what is known as a \emph{saddle point}: $\xx$ will be a minimum along the eigenvectors of $\HH_f(\xx)$ for which the eigenvalues are positive and a maximum along the eigenvectors of $\HH_f(\xx)$ for which the eigenvalues are negative.
  \end{remark}

\subsection{Characterization of convexity}
\boxdef{
For a convex set $S \subseteq R^n$, we say that a function $f:S \to
\mathbb{R}$ is  \textbf{strictly convex on $S$} if for any two points $\xx_1,\xx_2 \in S$ and any $\theta \in (0,1)$ we have that:
%
$$ f\left (\theta \xx_1 + (1-\theta)\xx_2 \right ) < \theta f (\xx_1) + \Big (1-\theta\Big )f(\xx_2).$$
%
}

\begin{theorem}
Let $S \subseteq \R^n$ be open and convex, and let $f:S \to \R$ be twice continuously differentiable.
\begin{enumerate}
\item If $H_{f}(\xx)$ is positive semi-definite for any $\xx \in S$ then $f$ is convex on $S$.
\item If $H_{f}(\xx)$ is positive definite for any $\xx \in S$ then $f$ is \textbf{strictly} convex on $S$.
\item If $f$ is convex, then $H_{f}(\xx)$ is positive semi-definite $\forall \xx \in S$.
\end{enumerate}
\end{theorem}

\begin{proof}
\hspace{0.1in}\newline
\begin{enumerate}
\item By applying Theorem~\ref{thm:taylorremainder2}, we find that for some $\zz \in [\xx,\yy]$:
%
$$f(\yy) = f(\xx) + \grad f(\xx)^\trp (\yy - \xx) + \frac{1}{2} \Big ( (\yy - \xx)^\trp H_{f}(\zz) (\yy - \xx)   \Big )$$
%
If $H_{f}(\zz)$ is positive semi-definite, this necessarily implies that:
%
$$f(\yy) \geq f(\xx) + \grad f(\xx)^\trp (\yy - \xx)$$
%
and from Theorem~\ref{thm:convexlb} we get that $f$ is convex.
%
\item if $H_{f}(\xx)$ is positive definite, we have that:
%
$$f(\yy) > f(\xx) + \grad f(\xx)^\trp (\yy - \xx).$$
%
Applying the same idea as in Theorem~\ref{thm:convexlb} we can show that in this case $f$ is \textbf{strictly} convex.

\item Let $f$ be a convex function.  For $\xx \in S$, and some small $\lambda >0$, for any $\mathbf{d} \in \mathbb{R}^n$ we have that $\xx + \lambda \mathbf{d} \in S$.  From the Taylor expansion of $f$ we get:
%
$$
f(\xx+\lambda \mathbf{d}) = f(\xx) + \lambda \grad f(\xx)^\trp\mathbf{d}
+ \frac{\lambda^2}{2} \mathbf{d}^\trp H_{f}(\xx)  \mathbf{d} +
o(\lambda^2 \norm{\dd}_2^2)
.
$$
%
From Lemma~\ref{thm:convexlb} we get that if $f$ is convex then:
%
$$f(\xx+\lambda\mathbf{d}) \geq f(\xx) + \lambda \grad f(\xx)^\trp \mathbf{d}.$$
%
Therefore, we have that for any $\mathbf{d} \in \mathbb{R}^{n}$:

$$\frac{\lambda^2}{2} \mathbf{d}^\trp H_{f}(\xx)  \mathbf{d} + o(|| \lambda \mathbf{d}||^2) \geq 0$$

Dividing by $\lambda^2$ and taking $\lambda \to 0^+$ gives us that for any $\mathbf{d} \in \mathbb{R}^n$: $\mathbf{d}^\trp H_{f}(\xx)  \mathbf{d} \geq 0$.\qedhere
\end{enumerate}
\end{proof}

\begin{remark}
It is important to note that if $S$ is open and $f$ is strictly
convex, then $\HH_{f}(\xx)$ may still (only) be positive semi-definite
$\forall \xx \in S$. Consider $f(x) = x^4$ which is strictly convex,
then the Hessian is $\HH_{f}(x) = 12x^2$ which equals 0 at $x = 0$.
\end{remark}

\section{Gradient Descent - An Approach to Optimization?}

We have begun to develop an understanding of convex functions, and
what we have learned already suggests a way for us to try to find an
approximate minimizer of a given convex function.

Suppose $f : \R^n \to \R$ is convex and differentiable, and we want to
solve
\begin{align*}
\min_{\xx \in \R^n} & \ f(\xx)
\end{align*}

We would like to find $\xx^*$, a global minimizer of $f$.
Suppose we start with some initial guess  $\xx_0$,
and we want to update it to $\xx_1$ with $f(\xx_1) < f(\xx_0)$.
If we can repeatedly make updates like this, maybe we eventually
find a point with nearly minimum function value, i.e. some $\xxtil$ with $f(\xxtil) \approx f(\xx^*)$?

Recall that $f(\xx_0+\ddelta) = f(\xx_0) + \grad f(\xx_0)^\trp\ddelta +
o(\norm{\ddelta}_2)$.
This means that if we choose
${\xx_1 = \xx_0 - \lambda \grad f(\xx_0)}$, we get
\[
  f(\xx_0 - \lambda \grad f(\xx_0)) =
  f(\xx_0) -\lambda \norm{\grad f(\xx_0)}_2^2 + o(\lambda \norm{\grad f(\xx_0)}_2)
\]
And because $f$ is convex, we know that $\grad f(\xx_0) \neq \veczero$
unless we are already at a global minimum.
So, for some small enough $\lambda > 0$, we should get $f(\xx_1) <
f(\xx_0)$ unless we're already at a global minimizer.
This idea of taking a step in the direction of $-\grad f(\xx_0)$ is
what is called \emph{Gradient Descent}.
% In fact, with some more work, we can argue that if we repeat this
% procedure, we will get a sequence $\xx_1, \xx_2, \xx_3, \ldots$ s.t.
% $f(\xx_i) \to f(\xx^*)$.\todo{do that? how does this square with the
%   pathological Ford-Fulkerson!?}
But how do we choose $\lambda$ each time? And
does this lead to an algorithm that quickly reaches a point with close to minimal function value?
To get good answers to these questions, we need to assume more about
the function $f$ that we are trying to minimize.

In the following subsection, we will see some conditions that
suffice.
But there are also many other settings where one can show that some
form of gradient descent converges.

\subsection{A Quantitative Bound on Changes in the Gradient}
\boxdef{
  Let $f: S \to \R$ be a differentiable
  function, where $S\subseteq \R^n$ is convex and open.
  We say that $f$ is $\beta$\emph{-gradient Lipschitz} iff for all $\xx,
  \yy \in S$
  \[
    \norm{\grad f(\xx) - \grad f(\yy)}_2 \leq \beta \norm{\xx-\yy}_2
    .
  \]
  We also refer to this as $f$ being $\beta$\emph{-smooth}.
}

\begin{proposition}
\label{prp:hessiangradlip}
  Consider a twice continuously differentiable $f: S \to \R$.
  Then $f$ is $\beta$-gradient Lipschitz if and only if for all $\xx \in
 S$,
  $\norm{\HH_f(\xx)} \leq \beta$.
\end{proposition}
 You will prove this in Exercise 1 of this week's exercises.
% \begin{proof}
%   We leave the proof for Section/Exercises.
%   \todo{phrasing and write-up}
% \end{proof}

\begin{proposition}
  \label{prp:lipgradub}
  Let $f: S \to \R$ be a $\beta$-gradient Lipschitz function.
  Then for all $\xx, \yy \in S$,
  % such that $[\xx,\yy] \in S$,
  \[
    f(\yy) \leq f(\xx) + \grad f(\xx)^\trp (\yy-\xx) + \frac{\beta}{2}
    \norm{\xx-\yy}_2^2
    \]
\end{proposition}

To prove this proposition, we need the following result from
multi-variate calculus. This is a restricted form of the fundamental
theorem of calculus for line integrals.

\begin{proposition}
  Let $f: S \to \R$ be a differentiable function, and consider
  $\xx, \yy$ such that $[\xx,\yy] \in S$.
  Let ${\xx_{\theta} = \xx + \theta (\yy - \xx)}$.
  Then
  \[
    f(\yy) = f(\xx) + \int_{\theta = 0}^1 \grad f(\xx_{\theta})^\trp (\yy-\xx) d\theta
    \]
\end{proposition}
Now, we're in a position to show Proposition~\ref{prp:lipgradub}
\begin{proof}[Proof of Proposition~\ref{prp:lipgradub}]
  Let $f: S \to \R$ be a $\beta$-gradient Lipschitz function.
  Consider arbitrary $\xx, \yy \in S$ such that $[\xx,\yy] \in S$

  \begin{align*}
    f(\yy)
    &= f(\xx) + \int_{\theta = 0}^1 \grad
      f(\xx_{\theta})^\trp(\yy-\xx) d\theta
    \\
    &= f(\xx)
     + \int_{\theta = 0}^1 \grad
      f(\xx)^\trp(\yy-\xx) d\theta
      + \int_{\theta = 0}^1
      (\grad f(\xx_{\theta}) - \grad f(\xx ))^\trp(\yy-\xx) d\theta
   \\
    &         \tag*{Next we use Cauchy-Schwarz pointwise.}
 \\
    &\tag*{ We also evaluate the
      first integral.}
    \\
    &\leq f(\xx)
     + \grad f(\xx)^\trp(\yy-\xx)
      + \int_{\theta = 0}^1
      \norm{ \grad f(\xx_{\theta}) - \grad f(\xx ) } \norm{\yy-\xx}
      d\theta
    \\
    &\tag*{Then we apply $\beta$-gradient Lipschitz and note
      $\xx_\theta -\xx = \theta (\yy-\xx)$.}
    \\
    &\leq f(\xx)
     + \grad f(\xx)^\trp(\yy-\xx)
      + \int_{\theta = 0}^1
    \beta \theta \norm{\yy-\xx}^2
      d\theta
      .
      \\
    &= f(\xx)
     + \grad f(\xx)^\trp(\yy-\xx)
      + \frac{\beta}{2} \norm{\yy-\xx}^2
      .
  \end{align*}
\end{proof}

\subsection{Analyzing Gradient Descent}
It turns out that just knowing a function $f: \R^n \to \R$ is convex
and $\beta$-gradient Lipschitz is enough to let us figure out a reasonable
step size for Gradient Descent and let us analyze its convergence.

We start at a point $\xx_0 \in \R^n$, and we try to find a
point $\xx_1 = \xx_0 +\ddelta$ with lower function value.
We will let our upper bound from Proposition~\ref{prp:lipgradub} guide
us, in fact, we could ask, what is the \emph{best} update for
minimizing this upper bound, i.e. a $\ddelta$ solving
\[
  \min_{\ddelta \in \R^n}
  f(\xx_{0})
     + \grad f(\xx_{0})^\trp\ddelta
      + \frac{\beta}{2} \norm{\ddelta}^2
\]
We can compute the best according to this upper bound by noting
first that function is convex and continuously differentiable,
and hence will be minimized at any point where the gradient is zero.
Thus we want
$\veczero = \grad_{\ddelta} \left(
 f(\xx_{0})
     + \grad f(\xx_{0})^\trp\ddelta
     + \frac{\beta}{2} \norm{\ddelta}^2
\right)
=
\grad f(\xx_{0})
+
\beta \ddelta
$, which occurs at
$\ddelta = -\frac{1}{\beta}\grad f(\xx_{0})$.

Plugging in this value into the upper bound, we get that
$f(\xx_1) \leq f(\xx_0) - \nicefrac{\norm{\grad f(\xx_{0})}_2^2}{2\beta}$.

% \paragraph{The algorithm.}
Now, as our algorithm, we will start with some guess $\xx_0$, and then
at every step we will update our guess using the best step based on
our Proposition~\ref{prp:lipgradub} upper bound on $f$ at $\xx_i$, and so we get
\begin{align}
\label{eq:gdstep}
  \xx_{i+1} = \xx_{i}-\frac{1}{\beta}\grad f(\xx_{i})
  \text{ and }
    f(\xx_{i+1}) &\leq f(\xx_{i}) - \frac{\norm{\grad
                 f(\xx_{i})}_2^2}{2\beta}
  .
\end{align}
% Again, from the Proposition~\ref{prp:lipgradub} upper bound, we get
% \begin{align*}

%                  .
% \end{align*}

% \paragraph{Convergence analysis}
Let us try to prove that our algorithm converges toward an $\xx$ with low
function value.

We will measure this by looking at
\[
  \gap_i =  f(\xx_{i}) - f(\xx^*)
  \]
  where $\xx^*$ is a global minimizer of $f$ (note that there may not
  be a unique minimizer of $f$).
We will try to show that this function value gap grows small.
Using $f(\xx_{i+1})- f(\xx_{i}) =
  \gap_{i+1}- \gap_i $, we get
\begin{align}
\label{eq:gdgapchange}
 \gap_{i+1}- \gap_i &\leq - \frac{\norm{\grad
  f(\xx_{i})}_2^2}{2\beta}
\end{align}
% \begin{align*}
%   f(\xx_{i+1}) &\leq f(\xx_{i}) - \frac{\norm{\grad
%                  f(\xx_{i})}_2^2}{2\beta}
%   \\
% \tag*{Use $f(\xx_{i+1})- f(\xx_{i}) =
%   \gap_{i+1}- \gap_i $}
%   \\
%  \gap_{i+1}- \gap_i &\leq - \frac{\norm{\grad
%   f(\xx_{i})}_2^2}{2\beta}
% \end{align*}
If the $\gap_i$  value is never too much bigger than $\frac{\norm{\grad
  f(\xx_{i})}_2^2}{2\beta}$, then this should help us show we are making
progress.
But how much can they differ? We will now try to show a limit on this.

Recall that in the previous chapter we showed the following theorem.
\begin{theorem}\label{thm:convexlb}
Let  $S$ be an open convex subset of $\mathbb{R}^n$, and
let $f:S\to \mathbb{R}$ be a differentiable function.
Then, $f$ is convex if and only if for any $\xx,\yy \in S$ we have that $f(\yy) \geq f(\xx) + \grad f(\xx)^\trp  (\yy-\xx)$.
\end{theorem}


Using the convexity of $f$ and the lower bound on
convex functions given by Theorem~\ref{thm:convexlb}, we have that
\begin{align}
\label{eq:basiciterlb}
f(\xx^*) \geq f(\xx_i) + \grad f(\xx_{i})^\trp (\xx^* - \xx_i)
\end{align}
Rearranging gets us
\begin{align}
\label{eq:convexubgapsmooth}
  \gap_i &\leq \grad f(\xx_{i})^\trp (\xx_i - \xx^*)
  \\
 & \leq \norm{\grad f(\xx_{i})}_2 \norm{\xx_i - \xx^*}_2
            \tag*{ by Cauchy-Schwarz.}
\end{align}
At this point, we are essentially ready to connect
Equation~\eqref{eq:gdgapchange} with Equation~\eqref{eq:convexubgapsmooth}
and analyze the convergence rate of our algorithm.

However, at the moment, we see that the change $\gap_{i+1}- \gap_i$
in how close we are to the optimum function value is governed by the norm of the gradient $\norm{\grad
  f(\xx_{i})}_2$, while the size of the gap is related to \emph{both}
this quantity and the distance $\norm{\xx_i - \xx^*}_2$ between the
current solution $\xx_i$ and an optimum $\xx^*$.
Do we need both or can we get rid of, say, the distance?
Unfortunately, with this algorithm and for this class of functions, a
dependence on the distance is necessary.
However, we can simplify things considerably using the following
observation, which you will prove in the exercises (Exercise 2):
\begin{claim}
  \label{clm:gdoptdist}
  When running Gradient Descent as given by the step in
  Equation~\eqref{eq:gdstep},  for all $i$
  $\norm{\xx_i - \xx^*}_2 \leq \norm{\xx_0 - \xx^*}_2$.
\end{claim}
Combining this Claim with Equation~\eqref{eq:gdgapchange} and Equation~\eqref{eq:convexubgapsmooth},
\begin{align}
\label{eq:convexgapsmooth}
  \gap_{i+1}- \gap_i
  &\leq
    -
    \frac{1}{2\beta}
    \cdot
    \left(
    \frac{\gap_i}{\norm{\xx_0 - \xx^*}_2 }
    \right)^2
\end{align}
At this point, a simple induction will complete the proof of following result.
\begin{theorem}
  \label{thm:gdsmoothconv}
  Let $f : \R^n \to \R$ be a $\beta$-gradient Lipschitz, convex
  function.
  Let $\xx_0$ be a given starting point,
  and let $\xx^* \in \argmin_{\xx \in \R^n} f(\xx)$ be a minimizer of $f$.
 The Gradient Descent algorithm given by
  \[
   \xx_{i+1} = \xx_{i} - \frac{1}{\beta} \grad f(\xx_{i})
 \]
 ensures that the $k$th iterate satisfies
 \[
   f(\xx_{k}) - f(\xx^*)\leq \frac{2\beta \norm{\xx_0 - \xx^*}_2^2}{k+1}
   .
   \]
\end{theorem}
Carrying out this induction is one of the Week 2 exercises (Exercise 3).

\section{Accelerated Gradient Descent}


It turns out that we can get an algorithm that converges substantially
faster than Gradient Descent, using an approach known as
\emph{Accelerated Gradient Descent}, which was developed by
Nesterov~\cite{n83}.
This algorithm in turn improved on some earlier results by Nemirovski and Yudin~\cite{ny83}.
The phenomenon of acceleration was perhaps first understood in the context of
quadratic functions, minimizing $\xx^\trp \AA \xx - \xx^\trp \bb$
when $\AA$ is positive definite -- for this case, the Conjugate
Gradient algorithm was developed independently by Hestenes and
Stiefel~\cite{hs52} (here at ETH!), and by Lanczos~\cite{l52}.
In the past few years, providing more intuitive explanations of
acceleration has been a popular research topic.
This presentation is based on an analysis of Nesterov's algorithm
developed by Diakonikolas and Orecchia~\cite{do19}.

% \todo{picture? discuss momentum?}

We will adopt a slightly different approach to analyzing this
algorithm than what we used in the previous section for Gradient Descent.

We will use $\xx_0$ to denote the starting point of our algorithm, and
we will produce a sequence of iterates $\xx_0, \xx_1, \xx_2, \ldots,
\xx_k $.
At each iterate $\xx_{i}$, we will compute the gradient $\grad
f(\xx_{i})$.
However, the way we choose  $\xx_{i+1}$ based on what we know so far
will now be a little more involved than what we did for Gradient Descent.

To help us understand the algorithm, we are going to introduce two
more sequences of iterates $\yy_0, \yy_1, \yy_2, \ldots,
\yy_k \in \R^n $ and $\vv_0, \vv_1, \vv_2, \ldots,
\vv_k \in \R^n$.

The sequence of $\yy_i$'s will be constructed to help us get as low a
function value as possible at $f(\yy_i)$, which we will consider our
current solution and the last iterate $\yy_k$ will be the output solution of
our algorithm.

The sequence of $\vv_i$'s will be constructed to help us get a lower
bound on $f(\xx^*)$.

By combining the upper bound on the function value of our current
solution $f(\yy_i)$ with a lower bound on the function value at an
optimal solution $f(\xx^*)$, we get an upper bound on the gap
 $f(\yy_i)-f(\xx^*)$
between
the value of our solution and the optimal one.
% We denote the gap by
% $\gap_i = f(\yy_i)-f(\xx^*)$.
Finally, each iterate $\xx_i$, which will be where we evaluate
gradient $\grad f(\xx_{i})$, is chosen through a trade-off between
wanting to reduce the upper bound and wanting to increase the lower bound.


\paragraph{The upper bound sequence: $\yy_i$'s.}
The point $\yy_i$ will be chosen from $\xx_i$ to minimize an upper
bound on $f$ based at $\xx_i$.
This is similar to what we did in the previous section.
We let $\yy_i = \xx_i +\ddelta_i$
and use choose $\ddelta_i$ to minimize the upper bound
$ \ff(\yy_i) \leq f(\xx_{i})  + \grad f(\xx_{i})^\trp\ddelta_i
      + \frac{\beta}{2} \norm{\ddelta_i}^2$, which gives us
\[
  \yy_i = \xx_{i}-\frac{1}{\beta}\grad f(\xx_{i})
  \text{ and }
    f(\yy_i) \leq f(\xx_{i}) - \frac{\norm{\grad
                 f(\xx_{i})}_2^2}{2\beta}
  .
\]
We will introduce a notation for this upper bound

\begin{equation}
  U_i = f(\yy_i) \leq  f(\xx_{i})
  - \frac{\norm{\grad  f(\xx_{i})}_2^2}{2\beta}
  .\label{eq:agdub}
\end{equation}
\paragraph{Philosophizing about lower bounds\protect\footnote{YMMV. People
    have a lot of different opionions about how to understand acceleration, and you should take my thoughts with a grain of salt.}.}
A crucial ingredient to establishing an upper bound on $\gap_i $ was a
lower bound on $f(\xx^*)$.

In our analysis of Gradient Descent, in Equation~\eqref{eq:convexubgapsmooth}, we used the
lower bound ${f(\xx^*) \geq f(\xx_i) - \norm{\grad f(\xx_{i})}_2 \norm{\xx_i - \xx^*}_2 }$.
%
We can think of the Gradient Descent analysis as being based on a
tension between two statements: Firstly that ``a large gradient implies we quickly approach
the optimum'' and secondly ``the function value gap to optimum cannot exceed the magnitude of
the current gradient (scaled by distance to opt)''.

This analysis does not use that we have seen many different
function values and gradients, and each of these can be used to
construct a lower bound on the optimum value $f(\xx^*)$, and, in
particular, it is not clear that the last gradient provides the best bound.
To do better, we will try to use lower bounds that take advantage of
all the gradients we have seen.

\boxdef{
We will adopt a new notation for inner products that sometimes is more
convenient when dealing with large expressions:
$\ip{\aa,\bb} \defeq \aa^\trp \bb $.
}
%
\paragraph{The lower bound sequence: $\vv_i$'s.}
We can introduce weights $a_i > 0$ for each step and combine the
gradients we have observed into one lower bound based on a weighted
average. Let us use $A_i = \sum_{j \leq i} a_j$ to denote the sum of
the weights. Now a general lower bound on the function value at any
$\vv \in \R^n$ is
:
\[
  f(\vv)
  \geq
  \frac{1}{A_i}
  \sum_{j \leq i} a_j \left(  f(\xx_j)
    + \ip{\grad
      f(\xx_j) , \vv - \xx_j}
    \right)
  \]
 However, to use Cauchy-Schwarz on each individual term here to
 instantiate this bound at $\xx^*$ does not give us anything useful.
 Instead, we will employ a somewhat magical trick: we introduce a
 regularization term
 \[
   \phi(\vv) \defeq \frac{\sigma}{2} \norm{ \vv -
     \xx_0}_2^2
   .
 \]
 We will choose the value $\sigma > 0$ later.
 Now  we derive our lower bound $L_i$
 \begin{align*}
  f(\xx^*)
  &\geq
    \frac{1}{A_i}
     \left(
\phi(\xx^*)
  +
  \sum_{j \leq i} a_j  f(\xx_j)
    + \ip{ a_j\grad
      f(\xx_j) , \xx^* - \xx_j}
 \right)
  -
 \frac{\phi(\xx^*)}{A_i}
    \\
   &\geq
     \min_{\vv \in \R^n} \setof{
    \frac{1}{A_i}
     \left(
\phi(\vv)
  +
  \sum_{j \leq i} a_j  f(\xx_j)
    + \ip{ a_j\grad
      f(\xx_{j}) ,\vv - \xx_j}
  \right)
     }
     -
     \frac{\phi(\xx^*)}{A_i}
   \\
   &= L_i
 \end{align*}
 We will let $\vv_i$ be the $\vv$ obtaining the minimum in the
 optimization problem appearing in the definition of $L_i$, so that
 \[
   L_i
   =
   \frac{1}{A_i}
   \left(
     \phi(\vv_i)
     +
     \sum_{j \leq i} a_i  f(\xx_i)
     + \ip{ a_i\grad
       f(\xx_{i}) ,\vv_i - \xx_i}
   \right)
   -
   \frac{\phi(\xx^*)}{A_i}
\]
\paragraph{How we will measure convergence.}
We have designed the upper bound $U_i$ and the lower bound $L_i$ such
that $\gap_i = f(\yy_i)-f(\xx^*) \leq U_i - L_i$.

As you will show in Exercise~3, we can prove the
convergence of Gradient Descent directly by an induction that
establishes $1/\gap_i \leq C \cdot i $ for some constant $C$ depending
on the Lipschitz gradient parameter $\beta$ and the distance $\norm{\xx_0
  - \xx^*}_2$.

To analyze Accelerated Gradient Descent, we will adopt a similar, but
slightly different strategy, namely trying to show that $(U_i - L_i) r(i)$
is non-increasing for some positive ``rate function'' $r(i)$. Ideally $r(i)$ should
grow quickly, which would imply that $\gap_i$ quickly gets small.
We will also need to show that $(U_0 - L_0) r(0) \leq C$ for some constant
$C$ again depending on $\beta$ and $\norm{\xx_0 - \xx^*}_2$.
%
Then, we'll be able to conclude that
\[
\gap_i \cdot r(i) \leq (U_i - L_i) r(i) \leq (U_{i-1} - L_{i-1}) r(i-1) \leq
\cdots \leq (U_0 - L_0) r(0) \leq C,
\]
and hence $\gap_i \leq C/r(i)$.

This framework is fairly general. We could have also used it to
analyze Gradient Descent, and it works for many other optimization
algorithms too.

We are going to choose our rate function $r(i)$ to be exactly $A_i$,
which of course is no accident!
As we will see, this interacts nicely with our lower
bound $L_i$.
Hence, our goals are to
\begin{enumerate}
\item provide an upper bound on $A_0 (U_0 - L_0)$,
\item and show that $A_{i+1} (U_{i+1} - L_{i+1})  \leq A_{i} (U_{i} - L_{i}) $,
\end{enumerate}

\paragraph{Establishing the convergence rate.}
Let's start by looking at the change in the upper bound scaled by our
rate function:

\begin{align}
  \label{eq:ubgap}
  A_{i+1}U_{i+1} - A_i U_i
  =&
  A_{i+1}\left( f(\yy_{i+1}) - f(\xx_{i+1}) \right)
  -
  A_{i}\left( f(\yy_{i}) - f(\xx_{i+1}) \right)
  +
  (A_{i+1}-A_{i}) f(\xx_{i+1})
\\
\leq &
     A_{i+1}\left(
     - \frac{\norm{\grad  f(\xx_{i+1})}_2^2}{2\beta}
       \right)
       & \tag*{First term controlled by Equation~\eqref{eq:agdub}.}
  \\
  &+
     A_{i}
    \ip{\grad f(\xx_{i+1}), \yy_{i} - \xx_{i+1}  }
    \tag*{Second term bounded by Theorem~\ref{thm:convexlb}.}
  \\
   &+ a_{i+1} f(\xx_{i+1})
     \tag*{Third term uses $a_{i+1} = A_{i+1}-A_{i}$.}
\end{align}

The solution $\vv_i$ to the minimization in the lower bound $L_i$
turns out to be relatively simple to characterize.
By using derivatives to find the optimum, we first analyze the initial
value of the lower bound $L_0$.

\newpage

 \begin{claim}
   \label{clm:lbinit}
   \noindent
   \begin{enumerate}
   \item $\vv_0 = \xx_0 - \frac{a_0}{\sigma} \grad f(\xx_0)$
   \item
     \label{clm:lbinit:value}
     $L_0 = f(\xx_0) - \frac{a_0}{2\sigma} \norm{\grad f(\xx_0)
     }_2^2 - \frac{\sigma }{2 a_0}\norm{ \xx^* - \xx_0}_2^2 $.
\end{enumerate}
\end{claim}
You will prove Claim~\ref{clm:lbinit} in the exercises for Week 3.
Noting $A_0 = a_0$,
we see from Equation~\eqref{eq:agdub} and Part~\ref{clm:lbinit:value}
of Claim~\ref{clm:lbinit}, that
\begin{align}
  \label{eq:gapinit}
  A_0( U_0 - L_0 ) \leq
  \left(
  \frac{a_0^2}{2\sigma}
  -
    \frac{a_0}{2\beta}
  \right)
  \norm{\grad f(\xx_0)}_2^2
    +
    \frac{\sigma }{2}\norm{ \xx^* - \xx_0}_2^2
  % = \frac{\beta }{2}\norm{ \xx^* - \xx_0}_2^2
\end{align}
It will be convenient to introduce notation for the
rescaled lower bound $A_iL_i$ \emph{without} optimizing over $\vv$.
\[
  m_i(\vv) =
  \phi(\vv)
   - \phi(\xx^*)
  +
  \sum_{j \leq i} a_j  f(\xx_j)
    + \ip{ a_j\grad
      f(\xx_{j}) ,\vv - \xx_j}
  \]
Thus $A_iL_i-A_{i+1}L_{i+1} = m_{i}(\vv_i) - m_{i+1}(\vv)$.
Now, it is not too hard to show the
following relationships.
\begin{claim}
  \label{clm:lbstep}
   \noindent
   \begin{enumerate}
   \item
     \label{enu:lbaroundopt}
     $ m_{i}(\vv) = m_{i}(\vv_i) + \frac{\sigma}{2} \norm{ \vv -
   \vv_i}_2^2$
   \item  $m_{i+1}(\vv) = m_{i}(\vv)
     + a_{i+1}  f(\xx_{i+1})
     + \ip{ a_{i+1}\grad
      f(\xx_{i+1}) ,\vv - \xx_{i+1}}$
  \item  $\vv_{i+1} = \vv_i -  \frac{a_{i+1}}{\sigma} \grad f(\xx_{i+1})$
   \end{enumerate}
 \end{claim}
And again, you will prove Claim~\ref{clm:lbstep} in the exercises for
Week 3.
\emph{Hint for Part~\ref{enu:lbaroundopt}: note that $m_i(v)$ is a quadratic function, miminimized at $v_i$ and
  its Hessian equals $\sigma I$ at all $v$.}

Given Claim~\ref{clm:lbstep}, we see that
\begin{align}
  \nonumber
 A_iL_i-A_{i+1}L_{i+1}
\\
 &= m_{i}(\vv_i)  - m_{i+1}(\vv_{i+1})
 \\
 &=
-  a_{i+1}  f(\xx_{i+1})
-  \ip{ a_{i+1}\grad
  f(\xx_{i+1}) ,\vv_{i+1} - \xx_{i+1}}
-
\frac{\sigma}{2} \norm{ \vv_{i+1} -
   \vv_i}_2^2
\\
 &=
 -  a_{i+1}  f(\xx_{i+1})
-  \ip{ a_{i+1}\grad f(\xx_{i+1}) ,\vv_{i} - \xx_{i+1}}
+
   \frac{a_{i+1}^2}{2\sigma} \norm{ \grad f(\xx_{i+1})}_2^2
   \label{eq:lbgap}
\end{align}
This means that by combining Equation~\eqref{eq:ubgap}
and~\eqref{eq:lbgap} we get
\begin{align*}
  \
  A_{i+1}( U_{i+1} - L_{i+1})
 -
  A_{i}( U_{i} - L_{i})
  \leq
  \left(
  \frac{-A_{i+1}}{2\beta}
  +
\frac{a_{i+1}^2}{2\sigma}
  \right)
  \norm{ \grad f(\xx_{i+1})}_2^2
  %
  \\
  +  \ip{\grad f(\xx_{i+1}) ,
  A_{i+1} \xx_{i+1}
  - a_{i+1} \vv_{i}
  - A_{i} \yy_{i}}
  .
\end{align*}
%, again mainly using derivatives to characterize the optimum.

Now, this means that $ A_{i+1}( U_{i+1} - L_{i+1}) - A_{i}( U_{i} -
L_{i}) \leq 0$ if
\[
   A_{i+1} \xx_{i+1}
  - a_{i+1} \vv_{i}
  - A_{i} \yy_{i} = \veczero
\text{ and }
A_{i+1}/\beta \geq  a_{i+1}^2/\sigma
\]

We can get this by letting
$\xx_{i+1} = \frac{
A_{i} \yy_{i} + a_{i+1} \vv_{i}
}{A_{i+1}}
$, and $\sigma = \beta$ and $a_i = \frac{i+1}{2}$,
which implies that $A_i = \frac{(i+1)(i+2)}{4} > a_i^2$.



By Equation~\eqref{eq:gapinit}, these parameter choices also imply
that
\[
  A_0 (U_0 - L_0) \leq \frac{\beta }{2}\norm{ \xx^* - \xx_0}_2^2  .
\]
Finally, by induction, we get
$ A_{i}( U_{i} - L_{i})  \leq  \frac{\beta }{2}\norm{ \xx^* -
  \xx_0}_2^2$.
Dividing through by $A_i$ and using $\gap_i \leq  U_{i} - L_{i}$
results in the following theorem.
\begin{theorem}
  Let $f : \R^n \to \R$ be a $\beta$-gradient Lipschitz, convex
  function.
  Let $\xx_0$ be a given starting point,
  and let $\xx^* \in \argmin_{\xx \in \R^n} f(\xx)$ be a minimizer of
  $f$.

The Accelerated Gradient Descent algorithm given by
\begin{align*}
  a_i &= \frac{i+1}{2}, A_i = \frac{(i+1)(i+2)}{4}     \\
 \vv_0 & = \xx_0 - \frac{1}{2\beta} \grad f(\xx_0) \\
  \yy_{i} &= \xx_{i} - \frac{1}{\beta} \grad f(\xx_{i}) \\
  \xx_{i+1} &=  \frac{
A_{i} \yy_{i} + a_{i+1} \vv_{i}
            }{A_{i+1}} \\
  \vv_{i+1} &= \vv_{i}  - \frac{a_{i+1}}{\beta} \grad f(\xx_{i+1})
\end{align*}
 ensures that the $k$th iterate satisfies
 \[
   f(\xx_{k}) - f(\xx^*)\leq \frac{4 \beta \norm{\xx_0 - \xx^*}_2^2}{(k+1)(k+2)}
   .
 \]
\end{theorem}
%   Suppose that $\xx$ is a suboptimal feasible point, and $\xx^*$ is an
% optimal feasible point, i.e. both satisfy the constraints and $f(\xx)
% > f(\xx^*)$.
% If we consider the set of candidate solutions
% $\xx_{\theta} = (1-\theta)\xx + \theta \xx^*$ for $\theta \in [0,1]$,
% the whole segment segment consists of feasible points,
% and the value of the function is decreasing as $\theta$ increases.
% So there is a at least one direction away from $\xx$ that consists of
% feasible points and gives and improvement in function value.

% If we start with a suboptimal point $\xx$
%   and move in the direction of an optimal point $\xx^*$, the function
%   value must improve along that direction.


%   Another important reason is that
%    if $g(\xx)$ is a convex function, then $\setof{\xx : \gg(\xx)
%     \leq b}$ is a convex set. And the intersection of a collection of
%   convex sets is convex.

%  If we go back to our optimization framework
% \begin{align*}
% \min_{\xx \in \R^d}  f(\xx)\\
% s.t. & \ g_1(\xx) \leq b_1 \\
% 	& \ . \\
% 	& \ . \\
% 	& \ . \\
% 	& \ g_m(\xx) \leq b_m
% %

% \end{align*}
% we can see on the basis of that the set of feasible points is convex.

% Suppose that $\xx$ is a suboptimal feasible point, and $\xx^*$ is an
% optimal feasible point, i.e. both satisfy the constraints and $f(\xx)
% > f(\xx^*)$.
% If we consider the set of candidate solutions
% $\xx_{\theta} = (1-\theta)\xx + \theta \xx^*$ for $\theta \in [0,1]$,
% the whole segment segment consists of feasible points,
% and the value of the function is decreasing as $\theta$ increases.
% So there is a at least one direction away from $\xx$ that consists of
% feasible points and gives and improvement in function value.


\FloatBarrier


%\bibliographystyle{alpha}
%\bibliography{refs}




%\end{document}

%%% Local Variables:
%%% mode: latex
%%% TeX-master: "agao21_script"
%%% TeX-engine: luatex
%%% End: